\documentclass{article}

\usepackage[english]{babel}
\usepackage{amsmath}
\usepackage{amssymb}
\usepackage{amsthm}
\usepackage{physics} % for \abs{}
\usepackage{bm} % for \bm bold text
\usepackage[hidelinks]{hyperref}


\usepackage{graphicx}

\graphicspath{ {./figures/} }

\newcommand{\centring}{\centering}

\DeclareMathOperator{\im}{Im} %Image

\newcommand{\nat}{\mathbb{N}}
\newcommand{\re}{\mathbb{R}}
\newcommand{\rat}{\mathbb{Q}} %rationals

\newcommand{\transpose}{\intercal}  %why on earth is this thing called an `intercal'?
\newcommand{\del}{\partial}
\newcommand{\Rightleftarrow}{\Leftrightarrow}

\newcommand{\dx}[1][x]{\text{d} #1 }  %usage $\dx$ -> dx, or $\dx[V]$ -> dV
\newcommand{\ddx}[1][x]{\text{d}^2 #1 } %usage $\ddx$ -> d^2 x or $\ddx[y] -> d^2y

\newcommand{\floor}[1]{\lfloor #1 \rfloor}
\newcommand{\ceil}[1]{\lceil #1 \rceil}

\theoremstyle{plain}
\newtheorem{thm}{Theorem}[section]
\newtheorem{lem}[thm]{Lemma}
\newtheorem{prop}[thm]{Proposition}
\newtheorem{claim}{Claim}[thm]
\newtheorem{corr}{Corollary}[thm]
\newtheorem*{thm*}{Theorem}
\newtheorem*{lem*}{Lemma}
\newtheorem*{prop*}{Proposition}
\newtheorem*{claim*}{Claim}
\newtheorem{corr*}{Corollary}[thm]



\theoremstyle{definition}
\newtheorem{defn}{Definition}[subsection]
\newtheorem*{defn*}{Definition}

\theoremstyle{remark}
\newtheorem{rem}{Remark}
\newtheorem*{rem*}{Remark}
\newtheorem{example}{Example}
\newtheorem*{example*}{Example}
\newtheorem{exercise}{Exercise}
\newtheorem*{exercise*}{Exercise}

